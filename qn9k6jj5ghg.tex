Here, we formalize the concept of extracting energy from the environment as quickly and efficiently as possible. Assume a metabolic network that can convert an initial substrate to a final product. To extract energy quickly, the rate of conversion from substrate to product should be maximized.  To extract energy efficiently, the total free energy change for the overall reaction should be distributed as evenly as possible across each step of a reaction


For a metabolic network, we define the probability that a reaction $i$ will fire in the forward  direction ${\mathcal P_{+i}}$ by normalizing its forward thermodynamic driving force by the sum of all forward and backward driving forces in the metabolic network. Similarly, we define the probability that reaction $i$ will fire in the backward direction $\mathcal P_{-i}$ by normalizing its backward thermodynamic force by  the sum of all forward and backward driving forces in the metabolic network