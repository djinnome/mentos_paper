Here, we formalize the concept of extracting energy from the environment as quickly and efficiently as possible. Assume a metabolic network that can convert an initial substrate (energy source) to biomass. To extract energy as quickly as possible, the overall growth rate of the metabolic network should be maximized.   To extract energy from the environment as efficiently as possible, the total free energy change for the macroscopic reaction should be distributed as evenly as possible across each microscopic reaction in the pathway in order to prevent bottlenecks that require an exponential amount of enzyme to generate biomass.


For a metabolic network, we define the probability that a reaction $i$ will fire in the forward  direction ${\mathcal P_{+i}}$ by normalizing its forward thermodynamic driving force by the sum of all forward and backward driving forces in the metabolic network. Similarly, we define the probability that reaction $i$ will fire in the backward direction $\mathcal P_{-i}$ by normalizing its backward thermodynamic force by  the sum of all forward and backward driving forces in the metabolic network