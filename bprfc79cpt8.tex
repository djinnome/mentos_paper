\begin{eqnarray}
v & = k_+[A]^{\gamma_A}[B]^{\gamma_B} - k_-[C]^{\gamma_C}[D]^{\gamma_D}
\end{eqnarray}